\documentclass[a4paper,10pt,oneside]{memoir}

\setlrmarginsandblock{1.5in}{0.8in}{*}
\setulmarginsandblock{1in}{0.8in}{*}
\setmarginnotes{13pt}{1in}{\onelineskip}
\checkandfixthelayout

%%% Overall document settings
\usepackage{microtype}

\usepackage{enumitem}
\usepackage{tabularray}
\UseTblrLibrary{booktabs}

\usepackage{xcolor}

\usepackage{fontawesome5}
% \usepackage{worldflags}
\usepackage{soul}

% For madsen chapter style:
\usepackage{graphicx}

\usepackage[no-math]{fontspec}
\usepackage[hangul]{kotex}
\SingleSpacing

\usepackage{relsize}

%%% Font settings
\usepackage{libertine}
\setmonofont{JuliaMono}[
  Extension      = .ttf,
  UprightFont    = *-Regular,
  BoldFont       = *-Bold,
  ItalicFont     = *-RegularItalic,
  BoldItalicFont = *-BoldItalic,
  Scale          = MatchLowercase,
]
\defaultfontfeatures{RawFeature={+axis={wght=400}}}
\setmainhangulfont{Noto Serif CJK KR}[
  Scale         = MatchUppercase,
  BoldFont      = *,
  BoldFeatures  = {RawFeature={+axis={wght=700}}},
  AutoFakeSlant = 0.15,
  Renderer=OpenType,
]
\setsanshangulfont{KoPubWorldDotum_Pro}[
  Scale    = MatchUppercase,
  BoldFont = {* Bold},
]
\setmonohangulfont{D2Coding}[
  Scale = MatchLowercase,
]

\newfontfamily\redemonfont{tsukimirounded}[
  Extension = .ttf,
  Ligatures = TeX,
  Scale = MatchUppercase,
  UprightFont = *-Medium,
  BoldFont = *-Bold,
]

\newcommand\Rdemon{{\redemonfont ReDemon UI}}
\def\cplus{\raisebox{.4ex}{\relsize{-3}{\texttt{+}}}}
\def\C++{C\nolinebreak\hspace{-.081em}\cplus\nolinebreak\hspace{-0.02em}\cplus}

\setlength{\parindent}{0pt}

%%% Memoir settings
\thispagestyle{empty}
\setsecnumdepth{none}

\newcommand{\ruledsec}[1]{\large\scshape\raggedright\strut #1\hrule}
\setsecheadstyle{\ruledsec}
\setaftersecskip{.5\onelineskip}

\newcommand{\subsec}[1]{\bfseries\raggedright\strut #1}
\setsubsecheadstyle{\subsec}
\setaftersubsecskip{.2\onelineskip}

\newcommand*\mail[1]{\href{mailto:#1}{\texttt{#1}}}
\newcommand*\github[1]{\href{https://github.com/#1}{\texttt{#1}}}
\newcommand*\website{\href{https://ropas.snu.ac.kr/~jhlee/}{\texttt{ropas.snu.ac.kr/\textasciitilde jhlee}}}
\newcommand*\blog{\href{https://jaylee.xyz}{\texttt{jaylee.xyz}}}

\ExplSyntaxOn
\keys_define:nn { entry }
  {
    title .tl_set:N = \l_cv_title,
    position .tl_set:N = \l_cv_position,
    body .tl_set:N = \l_cv_body,
    note .tl_set:N = \l_cv_note,
    date .tl_set:N = \l_cv_date,
    paper .tl_set:N = \l_cv_paper,
    github .tl_set:N = \l_cv_github,
    video .tl_set:N = \l_cv_video,
    url .tl_set:N = \l_cv_url,
  }

\tl_new:N \l_meta
\NewDocumentCommand {\Entry} { m }
  {
    \group_begin: % keep the assignment to the keys local
    \keys_set:nn { entry } { #1 }

    \tl_clear:N \l_meta
    \tl_if_blank:VF \l_cv_paper
      { \tl_put_right:Nn \l_meta { ~\href{\l_cv_paper}{\color{black}\faFile*[regular]} } }
    \tl_if_blank:VF \l_cv_github
      { \tl_put_right:Nn \l_meta { ~\href{\l_cv_github}{\color{black}\faGithub} } }
    \tl_if_blank:VF \l_cv_video
      { \tl_put_right:Nn \l_meta { ~\href{\l_cv_video}{\color{black}\faYoutube} } }
    \tl_if_blank:VF \l_cv_url
      { \tl_put_right:Nn \l_meta { ~\href{\l_cv_url}{\color{black}\faGlobe} } }

    \sidepar
      {
        \footnotesize
        \tl_if_blank:VF \l_meta { \tl_use:N \l_meta \\ }
        \textsf{\tl_use:N \l_cv_date}
      }

    \tl_if_blank:VTF \l_cv_position
      { \textbf{\textsf{\tl_use:N \l_cv_title}} }
      { \textbf{\textsf{\tl_use:N \l_cv_title,~}} \textit{\tl_use:N \l_cv_position} }

    \tl_if_blank:VF \l_cv_body { \\ \tl_use:N \l_cv_body }

    \tl_if_blank:VF \l_cv_note { \\ \textit{\tl_use:N \l_cv_note} }

    \group_end:
  }
\ExplSyntaxOff

\usepackage[pdfencoding=auto,bookmarks]{hyperref}
\hypersetup{
  colorlinks,
  linkcolor  = {blue!73!black},
  citecolor  = {red!73!black},
  urlcolor   = {violet!73!black},
  pdflang    = {en},
  pdfsubject = {이재호의 이력서},
}

\begin{document}
{\huge 이재호~~\textsc{Jay Lee} \hfill 이력서}

\vspace{0.5\onelineskip}
\faEnvelope[regular]~~\mail{jaeho.lee@snu.ac.kr}, \mail{jhlee@ropas.snu.ac.kr} \\
\faGlobe~~\href{https://ropas.snu.ac.kr/~jhlee/}{\texttt{ropas.snu.ac.kr/\textasciitilde jhlee}}, \href{https://jaylee.xyz/}{\texttt{jaylee.xyz}} \\
\faGithub~~\github{Zeta611}

\section{연구 관심 분야}
프로그래밍 언어, 프로그램 분석, 함수형 프로그래밍, 인간-컴퓨터 상호작용(HCI)

\section{학력}
\Entry{
  date     = {2024/3--현재},
  title    = {서울대학교},
  position = {컴퓨터공학부 석사},
  body     = {지도교수: \href{http://kwangkeunyi.snu.ac.kr/}{이광근}},
}
% GPA: 4.2/4.3
\vspace{0.5\onelineskip}

\Entry{
  date     = {2018/3--2024/2},
  title    = {서울대학교},
  position = {전기·정보공학부 학사 (우등 졸업)},
  body     = {지도교수: \href{https://laser.snu.ac.kr/members/professor}{정윤찬}},
  note     = {2020--2021년 군 복무로 인한 휴학.},
}
% GPA: 3.81/4.3
\vspace{0.5\onelineskip}

\Entry{
  date     = {2015/3--2018/2},
  title    = {한국과학기술원 부설 한국과학영재학교 (KSA of KAIST)},
}
% GPA: 4.05/4.3


\section{주요 연구}
\Entry{
  date     = {OOPSLA 2025},
  title    = {\textsc{React-tRace:} A Semantics for Understanding React Hooks},
  body     = {\ul{Jay Lee}, Joongwon Ahn, Kwangkeun Yi},
  note     = {채택 (Accepted).},
  paper    = {https://doi.org/10.48550/arXiv.2507.05234},
  github   = {https://github.com/Zeta611/react-trace},
}
\vspace{0.5\onelineskip}

\Entry{
  date     = {UIST 2025\\Posters},
  title    = {\Rdemon: Reactive Synthesis by Demonstration for Web UI},
  body     = {\ul{Jay Lee}, Gyuhyeok Oh, Joongwon Ahn, Xiaokang Qiu},
  note     = {채택 (Accepted).},
  paper    = {https://doi.org/10.48550/arXiv.2507.10099},
  github   = {https://github.com/Zeta611/redemon-ui},
}
\vspace{0.5\onelineskip}

\Entry{
  date     = {PLDI 2025\\Student Research Competition (SRC)},
  title    = {Retargeting an Abstract Interpreter for a New Language by Partial Evaluation},
  body     = {\ul{Jay Lee}},
  note     = {대학원생 부문 2위 수상.},
  paper    = {https://doi.org/10.48550/arXiv.2507.04316},
  video    = {https://www.youtube.com/live/BRhBv_aYNks?t=5750},
  url      = {https://pldi25.sigplan.org/details/pldi-2025-src/1/},
}

\section{주요 수상 및 장학 내역}
\Entry{
  date     = {2025/6},
  title    = {PLDI 2025 Student Research Competition (SRC) 2위},
  position = {대학원생 부문},
  body     = {\textit{Retargeting an Abstract Interpreter for a New Language by Partial Evaluation} 연구에 대하여.},
}
\vspace{0.5\onelineskip}

\Entry{
  date     = {2024/9},
  title    = {우수 강의조교상},
  body     = {이광근 교수님의 \textit{SNU 4190.310 프로그래밍 언어} 과목에 대하여.},
}
\vspace{0.5\onelineskip}

\Entry{
  date     = {2018/3--2024/2},
  title    = {대통령과학장학금},
  position = {등록금 전액 및 학업장려금 지원},
  body     = {이공계 분야 최우수 학부생 120명에게 수여.},
}
\vspace{0.5\onelineskip}

\Entry{
  date     = {2017/5},
  title    = {Intel 국제과학기술경진대회 (ISEF) Finalist},
  position = {한국 대표},
  body     = {\textit{Receding Horizon Next-Best-View Planner Based Voronoi-Biased 3D Multi-Robot Exploration Algorithm} 프로젝트로 참가.},
}
\vspace{0.5\onelineskip}

\Entry{
  date     = {2015/3--2018/2},
  title    = {KSA 장학금},
  body     = {성적 우수 학생 대상.},
}


\section{교육 경력}
\subsection{조교}
\Entry{
  date     = {2025 봄},
  title    = {SNU 4190.664A 프로그램 분석},
}

\Entry{
  date     = {2025 봄},
  title    = {SNU 4190.310 프로그래밍 언어},
}

\Entry{
  date     = {2024 봄},
  title    = {SNU 4190.310 프로그래밍 언어},
  note     = {해당 과목으로 우수 강의조교상 수상.},
}

\Entry{
  date     = {2022 봄},
  position = {학부생 조교},
  title    = {SNU 4190.310 프로그래밍 언어},
}

\subsection{튜터링}
\Entry{
  date     = {2024 가을},
  title    = {서울대학교 SPLIT 기초 프로그래밍 튜터링},
  position = {파이썬 튜터},
}


\section{산업체 경력}
\Entry{
  date     = {2019/4--2019/12},
  title    = {정육각},
  position = {iOS 애플리케이션 개발자},
  body     = {신선육 유통 스타트업 정육각의 iOS 앱 개발.},
  url      = {https://jeongyookgak.com/},
}

\section{기타 경력}
\Entry{
  date     = {2025--현재},
  title    = {PL Reading Group\,@\,Seoul/Tokyo},
  position = {공동 운영진},
  body     = {서울대학교 프로그래밍 연구실과 도쿄과학대학 프로그래밍 연구 그룹이 함께하는 PL 스터디 그룹. 매주 화요일 오후 2시(KST/JST) 진행.},
}
\vspace{0.5\onelineskip}

\Entry{
  date     = {2025--현재},
  title    = {OCaml organization},
  position = {Opam 저장소 관리자},
  body     = {OCaml 패키지 저장소 관리.},
  github   = {https://github.com/orgs/ocaml/people?query=jay+lee},
}
\vspace{0.5\onelineskip}

\Entry{
  date     = {2023--현재},
  title    = {easyword.kr (쉬운 우리말 전문용어)},
  position = {디자이너 및 개발자},
  body     = {컴퓨터과학 전문용어의 한국어 번역을 제안하고 논의하는 플랫폼. 이광근 교수님 주도, 한국정보과학회 후원.},
  url      = {https://easyword.kr/},
  github   = {https://github.com/Zeta611/easyword},
}
\vspace{0.5\onelineskip}

\Entry{
  date     = {2020--현재},
  title    = {simplebnf},
  position = {개발자},
  body     = {간단한 DSL을 통해 BNF를 조판할 수 있는 \LaTeX{} 패키지. \href{https://ctan.org/pkg/simplebnf}{CTAN}에서 사용 가능.},
  github   = {https://github.com/Zeta611/simplebnf},
  url      = {https://ctan.org/pkg/simplebnf},
}
\vspace{0.5\onelineskip}

\Entry{
  date     = {2023},
  title    = {Cycloidal-Surfaces},
  position = {일러스트레이터 및 개발자},
  body     = {Asymptote를 이용한 매개화된 곡선 위의 사이클로이드 곡면 그림. Hyounggyu Choi. (2023) Invariance of the Area and Volume of Cycloid Surfaces and Trochoid Surfaces. \textit{The American Mathematical Monthly} 130:1, 49-62. 논문에 사용.},
  github   = {https://github.com/Zeta611/cycloidal-surfaces},
  paper    = {https://doi.org/10.1080/00029890.2022.2130677},
}
\vspace{0.5\onelineskip}

\Entry{
  date     = {2020},
  title    = {CycloidGen},
  position = {일러스트레이터 및 개발자},
  body     = {Ti\textit{k}Z와 Python을 이용한 매개화된 곡선 위의 사이클로이드 그림. Hyounggyu Choi. (2020) Invariance of the Length and the Area of Cycloids. \textit{The American Mathematical Monthly} 127:6, 537-544. 논문에 사용.},
  github   = {https://github.com/Zeta611/cycloidgen},
  paper    = {https://doi.org/10.1080/00029890.2020.1743611},
}
\vspace{0.5\onelineskip}

\Entry{
  date     = {2019--2024},
  title    = {한국텍학회 (KTS)},
  position = {회원},
  body     = {아래를 포함한 다수의 발표를 진행:
    \vspace{-0.5\onelineskip}
    \begin{itemize}[noitemsep]
      \item \href{https://githubcom/Zeta611/modern-tex-engines-automation-ai-latex-workshop-2023}{Modern \TeX: Engines, AI, and Automation}, 2023 KNU \LaTeX\ Workshop
      \item \href{https://github.com/Zeta611/tabularray-tutorial-latex-workshop-2022}{Drawing tables with \texttt{tabularray}}, 2022 KNU \LaTeX\ Workshop
      \item \href{https://github.com/Zeta611/asymptote-tutorial-latex-workshop-2021}{Asymptote: The Vector Graphics Language}, 2022 KNU \LaTeX\ Workshop
      \item \href{https://github.com/Zeta611/key-value-tutorial-latex-workshop-2021}{The ``key-value'' structure in \LaTeX}, 2021 KNU \LaTeX\ Workshop
      \item \href{https://github.com/Zeta611/beamer-tutorial-latex-workshop-2020}{\texttt{beamer}: Content-focused Presentation}, 2020 KNU \LaTeX\ Workshop
      \item \href{https://github.com/Zeta611/texnical-vim-kts-conf-2020}{\TeX{}nical Vim}, 2020 KTS Conference
      \item \href{https://github.com/Zeta611/chapterstyle-latex-workshop-2019}{\texttt{memoir}: chapter style}, 2019 KNU \LaTeX\ Workshop
    \end{itemize}
    \vspace{-0.5\onelineskip}
  },
}
\vspace{0.5\onelineskip}

\Entry{
  date     = {2018--2024},
  title    = {스누라이프 (SNULife)},
  position = {개발팀장},
  body     = {월 방문자 18만 명 이상의 서울대학교 웹 커뮤니티. 5천 명 이상이 사용한 시간표 계획 및 강의평 공유 iOS 앱 제작.},
  url      = {https://www.snulife.com/},
}
\vspace{0.5\onelineskip}

\Entry{
  date     = {2020--2021},
  title    = {대한민국 육군},
  position = {병장, 통신병},
  body     = {만기 전역.},
}
\vspace{0.5\onelineskip}

\Entry{
  date     = {2018},
  title    = {과대표},
  position = {전기·정보공학부},
  body     = {서울대학교 전기·정보공학부 과대표 선출.},
}
\vspace{0.5\onelineskip}

\Entry{
  date     = {KYPT 2016, 2017},
  title    = {한국청소년물리토너먼트},
  position = {팀장},
  body     = {고등학생 대상 물리 대회. 2017년 금상, 2016년 동상 수상팀의 팀장으로 활동.},
}
\vspace{0.5\onelineskip}

\Entry{
  date     = {KIPS 2016},
  title    = {Receding Horizon Next-Best-View Planner Based Voronoi-Biased 3D Multi-Robot Exploration Algorithm},
  body     = {\ul{J. Lee}, C. Lee, W. Jung, S. Song, S. Jo},
  note     = {국내 학술대회 논문.},
  paper    = {https://doi.org/10.3745/PKIPS.y2016m10a.579},
}
\vspace{0.5\onelineskip}

\Entry{
  date     = {2016},
  title    = {Frontiers Summer Program},
  position = {Worcester Polytechnic Institute (WPI)},
  body     = {미국 WPI에서 항공우주공학 및 심리학 과정 수료.},
}

\section{자연어}
\textbf{한국어}~(모국어, 1999), \textbf{영어/English}~(능숙, 2006), \textbf{스페인어/Español}~(초급, 2022)

\section{프로그래밍 언어}
C~(2012), \textbf{Python}~(2013), \C++~(2016), \textbf{\TeX}~(2016), Swift~(2018), Ti\textit{k}Z~(2018), \textbf{OCaml}~(2019), \lambda{} calculus~(2019), \textbf{\LaTeX3/expl3}~(2019), Asymptote~(2020), Scheme~(2020), AWK~(2020), JavaScript~(2020), Lua~(2021), \texttt{CWEB}~(2021), Yacc~(2021), \textbf{ReScript}~(2022), \textbf{React}~(2022), Rocq/Coq~(2023), Rust~(2023), \textbf{TypeScript}~(2023), Typst~(2024), Penrose~(2024), Nix~(2024), Lean~(2025).
\textbf{굵은 글씨}로 표기된 언어를 주로 사용.

\vfill\hfill \textsf{\today에~(NixOS와 macOS에서 Neovim과 \LaTeX3를 사용해) 작성함.}
\end{document}
