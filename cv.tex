\documentclass[a4paper,10pt,oneside]{memoir}

\setlrmarginsandblock{1.5in}{0.8in}{*}
\setulmarginsandblock{1in}{0.8in}{*}
\setmarginnotes{13pt}{1in}{\onelineskip}
\checkandfixthelayout

%%% Overall document settings
\usepackage{microtype}

\usepackage{enumitem}
\usepackage{tabularray}
\UseTblrLibrary{booktabs}

\usepackage{xcolor}

\usepackage{fontawesome5}
% \usepackage{worldflags}
\usepackage{soul}

% For madsen chapter style:
\usepackage{graphicx}

\usepackage[no-math]{fontspec}
\usepackage{kotex}

\usepackage{relsize}

%%% Font settings
\usepackage{libertine}
\setmonofont{JuliaMono}[
  Extension      = .ttf,
  UprightFont    = *-Regular,
  BoldFont       = *-Bold,
  ItalicFont     = *-RegularItalic,
  BoldItalicFont = *-BoldItalic,
  Scale          = MatchLowercase,
]
\defaultfontfeatures{RawFeature={+axis={wght=400}}}
\setmainhangulfont{Noto Serif CJK KR}[
  Scale         = MatchUppercase,
  BoldFont      = *,
  BoldFeatures  = {RawFeature={+axis={wght=700}}},
  AutoFakeSlant = 0.15,
  Renderer=OpenType,
]
\setsanshangulfont{KoPubWorldDotum_Pro}[
  Scale    = MatchUppercase,
  BoldFont = {* Bold},
]
\setmonohangulfont{D2Coding}[
  Scale = MatchLowercase,
]

\newfontfamily\redemonfont{tsukimirounded}[
  Extension = .ttf,
  Ligatures = TeX,
  Scale = MatchUppercase,
  UprightFont = *-Medium,
  BoldFont = *-Bold,
]

\newcommand\Rdemon{{\redemonfont ReDemon UI}}
\def\cplus{\raisebox{.4ex}{\relsize{-3}{\texttt{+}}}}
\def\C++{C\nolinebreak\hspace{-.081em}\cplus\nolinebreak\hspace{-0.02em}\cplus}

\setlength{\parindent}{0pt}

%%% Memoir settings
\thispagestyle{empty}
\setsecnumdepth{none}

\newcommand{\ruledsec}[1]{\large\scshape\raggedright\strut #1\hrule}
\setsecheadstyle{\ruledsec}
\setaftersecskip{.5\onelineskip}

\newcommand{\subsec}[1]{\bfseries\raggedright\strut #1}
\setsubsecheadstyle{\subsec}
\setaftersubsecskip{.2\onelineskip}

\newcommand*\mail[1]{\href{mailto:#1}{\texttt{#1}}}
\newcommand*\github[1]{\href{https://github.com/#1}{\texttt{#1}}}
\newcommand*\website{\href{https://ropas.snu.ac.kr/~jhlee/}{\texttt{ropas.snu.ac.kr/\textasciitilde jhlee}}}
\newcommand*\blog{\href{https://jaylee.xyz}{\texttt{jaylee.xyz}}}

\ExplSyntaxOn
\keys_define:nn { entry }
  {
    title .tl_set:N = \l_cv_title,
    position .tl_set:N = \l_cv_position,
    body .tl_set:N = \l_cv_body,
    note .tl_set:N = \l_cv_note,
    date .tl_set:N = \l_cv_date,
    paper .tl_set:N = \l_cv_paper,
    github .tl_set:N = \l_cv_github,
    video .tl_set:N = \l_cv_video,
    url .tl_set:N = \l_cv_url,
  }

\tl_new:N \l_meta
\NewDocumentCommand {\Entry} { m }
  {
    \group_begin: % keep the assignment to the keys local
    \keys_set:nn { entry } { #1 }

    \tl_clear:N \l_meta
    \tl_if_blank:VF \l_cv_paper
      { \tl_put_right:Nn \l_meta { ~\href{\l_cv_paper}{\color{black}\faFile*[regular]} } }
    \tl_if_blank:VF \l_cv_github
      { \tl_put_right:Nn \l_meta { ~\href{\l_cv_github}{\color{black}\faGithub} } }
    \tl_if_blank:VF \l_cv_video
      { \tl_put_right:Nn \l_meta { ~\href{\l_cv_video}{\color{black}\faYoutube} } }
    \tl_if_blank:VF \l_cv_url
      { \tl_put_right:Nn \l_meta { ~\href{\l_cv_url}{\color{black}\faGlobe} } }

    \sidepar
      {
        \footnotesize
        \tl_if_blank:VF \l_meta { \tl_use:N \l_meta \\ }
        \textsf{\tl_use:N \l_cv_date}
      }

    \tl_if_blank:VTF \l_cv_position
      { \textbf{\textsf{\tl_use:N \l_cv_title}} }
      { \textbf{\textsf{\tl_use:N \l_cv_title,~}} \textit{\tl_use:N \l_cv_position} }

    \tl_if_blank:VF \l_cv_body { \\ \tl_use:N \l_cv_body }

    \tl_if_blank:VF \l_cv_note { \\ \textit{\tl_use:N \l_cv_note} }

    \group_end:
  }
\ExplSyntaxOff

\usepackage[pdfencoding=auto,bookmarks]{hyperref}
\hypersetup{
  colorlinks,
  linkcolor  = {blue!73!black},
  citecolor  = {red!73!black},
  urlcolor   = {violet!73!black},
  pdflang    = {en},
  pdfsubject = {Jay Lee's Curriculum Vitae},
}

\begin{document}
{\huge \textsc{Jay Lee}~~이재호 \hfill \textsc{Curriculum Vitae}}

\vspace{0.5\onelineskip}
\faEnvelope[regular]~~\mail{jaeho.lee@snu.ac.kr}, \mail{jhlee@ropas.snu.ac.kr} \\
\faGlobe~~\href{https://ropas.snu.ac.kr/~jhlee/}{\texttt{ropas.snu.ac.kr/\textasciitilde jhlee}}, \href{https://jaylee.xyz/}{\texttt{jaylee.xyz}} \\
\faGithub~~\github{Zeta611}

\section{Research Interests}
Programming Languages, Program Analysis, Functional Programming, and Human-Computer Interaction

\section{Education}
\Entry{
  date     = {Mar 2024--Present},
  title    = {Seoul National University (SNU)},
  position = {M.S. in Computer Science and Engineering},
  body     = {Advised by \href{http://kwangkeunyi.snu.ac.kr/}{Kwangkeun Yi}.},
}
% GPA: 4.2/4.3
\vspace{0.5\onelineskip}

\Entry{
  date     = {Mar 2018--Feb 2024},
  title    = {Seoul National University (SNU)},
  position = {B.S. in Electrical and Computer Engineering (Cum laude)},
  body     = {Advised by \href{https://laser.snu.ac.kr/members/professor}{Yoonchan Jeong}.},
  note     = {Leave of absence for mandatory military service during 2020--2021.},
}
% GPA: 3.81/4.3
\vspace{0.5\onelineskip}

\Entry{
  date     = {Mar 2015--Feb 2018},
  title    = {Korea Science Academy of KAIST (KSA)},
  position = {High school for gifted students},
}
% GPA: 4.05/4.3


\section{Selected Research}
\Entry{
  date     = {OOPSLA 2025},
  title    = {\textsc{React-tRace:} A Semantics for Understanding React Hooks},
  body     = {\ul{Jay Lee}, Joongwon Ahn, Kwangkeun Yi},
  note     = {Conditionally accepted.},
  paper    = {https://doi.org/10.48550/arXiv.2507.05234},
  github   = {https://github.com/Zeta611/react-trace},
}
\vspace{0.5\onelineskip}

\Entry{
  date     = {UIST 2025\\Posters},
  title    = {\Rdemon: Reactive Synthesis by Demonstration for Web UI},
  body     = {\ul{Jay Lee}, Gyuhyeok Oh, Joongwon Ahn, Xiaokang Qiu},
  note     = {Submitted.},
  paper    = {https://doi.org/10.48550/arXiv.2507.10099},
  github   = {https://github.com/Zeta611/redemon-ui},
}
\vspace{0.5\onelineskip}

\Entry{
  date     = {PLDI 2025\\Student Research Competition (SRC)},
  title    = {Retargeting an Abstract Interpreter for a New Language by Partial Evaluation},
  body     = {\ul{Jay Lee}},
  note     = {Awarded 2\textsuperscript{nd} place in the graduate category.},
  paper    = {https://doi.org/10.48550/arXiv.2507.04316},
  video    = {https://www.youtube.com/live/BRhBv_aYNks?t=5750},
  url      = {https://pldi25.sigplan.org/details/pldi-2025-src/1/},
}

\section{Selected Honors}
\Entry{
  date     = {Jun 2025},
  title    = {PLDI 2025 Student Research Competition (SRC) 2\textsuperscript{nd} Place},
  position = {Graduate category},
  body     = {For the work on \textit{Retargeting an Abstract Interpreter for a New Language by Partial Evaluation}.},
}
\vspace{0.5\onelineskip}

\Entry{
  date     = {Sep 2024},
  title    = {Outstanding Teaching Assistant Award},
  body     = {For the course \textit{SNU 4190.310 Programming Languages} taught by Prof. Kwangkeun Yi.},
}
\vspace{0.5\onelineskip}

\Entry{
  date     = {Mar 2018--Feb 2024},
  title    = {National Presidential Scholarship for Science},
  position = {Full tuition waiver with an academic incentive},
  body     = {Offered to top-120 undergraduate students in the field of science and technology.},
}
\vspace{0.5\onelineskip}

\Entry{
  date     = {May 2017},
  title    = {Intel International Science and Engineering Fair (ISEF) Finalist},
  position = {Korean representative},
  body     = {For the project \textit{Receding Horizon Next-Best-View Planner Based Voronoi-Biased 3D Multi-Robot Exploration Algorithm}.},
}
\vspace{0.5\onelineskip}

\Entry{
  date     = {Mar 2015--Feb 2018},
  title    = {Korea Science Academy Fund Scholarship},
  body     = {For students with high GPAs.},
}


\section{Teaching}
\subsection{Teaching Assistant}
\Entry{
  date     = {Spring 2025},
  title    = {SNU 4190.664A Program Analysis},
}

\Entry{
  date     = {Spring 2025},
  title    = {SNU 4190.310 Programming Languages},
}

\Entry{
  date     = {Spring 2024},
  title    = {SNU 4190.310 Programming Languages},
  note     = {Awarded the Outstanding Teaching Assistant Award for this course.},
}

\Entry{
  date     = {Spring 2022},
  position = {Undergraduate TA},
  title    = {SNU 4190.310 Programming Languages},
}

\subsection{Tutoring}
\Entry{
  date     = {Fall 2024},
  title    = {SNU SPLIT Basic Programming Tutoring},
  position = {Python tutor},
}


\section{Industry Employment}
\Entry{
  date     = {Apr 2019--Dec 2019},
  title    = {Jeongyookgak},
  position = {iOS application developer},
  body     = {Developed an iOS app for Jeongyookgak, a distribution business startup that delivers fresh meat to customers.},
  url      = {https://jeongyookgak.com/},
}

\section{Miscellaneous}
\Entry{
  date     = {2025--Present},
  title    = {PL Reading Group\,@\,Seoul/Tokyo},
  position = {Co-organizer},
  body     = {PL reading group with participants from \textit{Programming Research Laboratory at Seoul National University} and \textit{Programming Research Group at the Institute of Science Tokyo.} Tuesday every week at 2PM KST/JST.},
}
\vspace{0.5\onelineskip}

\Entry{
  date     = {2025--Present},
  title    = {OCaml organization},
  position = {Opam repository maintainer},
  body     = {Maintaining the OCaml package repository.},
  github   = {https://github.com/orgs/ocaml/people?query=jay+lee},
}
\vspace{0.5\onelineskip}

\Entry{
  date     = {2023--Present},
  title    = {easyword.kr},
  position = {Designer \& developer},
  body     = {A platform to suggest and discuss Korean translations of computer science jargons. Led by Prof. Kwangkeun Yi and funded by the Korean Institute of Information Scientists and Engineers (KIISE).},
  url      = {https://easyword.kr/},
  github   = {https://github.com/Zeta611/easyword},
}
\vspace{0.5\onelineskip}

\Entry{
  date     = {2020--Present},
  title    = {simplebnf},
  position = {Developer},
  body     = {A \LaTeX{} package that provides a simple way to typeset BNF using a DSL. Available on \href{https://ctan.org/pkg/simplebnf}{CTAN}.},
  github   = {https://github.com/Zeta611/simplebnf},
  url      = {https://ctan.org/pkg/simplebnf},
}
\vspace{0.5\onelineskip}

\Entry{
  date     = {2023},
  title    = {Cycloidal-Surfaces},
  position = {Illustrator \& developer},
  body     = {Illustrations of cycloid surfaces on parametrized curves with Asymptote, used in Hyounggyu Choi. (2023) Invariance of the Area and Volume of Cycloid Surfaces and Trochoid Surfaces. \textit{The American Mathematical Monthly} 130:1, 49-62.},
  github   = {https://github.com/Zeta611/cycloidal-surfaces},
  paper    = {https://doi.org/10.1080/00029890.2022.2130677},
}
\vspace{0.5\onelineskip}

\Entry{
  date     = {2020},
  title    = {CycloidGen},
  position = {Illustrator \& developer},
  body     = {Illustrations of cycloids on parametrized curves with Ti\textit{k}Z and Python, used in Hyounggyu Choi. (2020) Invariance of the Length and the Area of Cycloids. \textit{The American Mathematical Monthly} 127:6, 537-544.},
  github   = {https://github.com/Zeta611/cycloidgen},
  paper    = {https://doi.org/10.1080/00029890.2020.1743611},
}
\vspace{0.5\onelineskip}

\Entry{
  date     = {2019--2024},
  title    = {Korean \TeX\ Society (KTS)},
  position = {Member},
  body     = {I have given a few talks including:
    \vspace{-0.5\onelineskip}
    \begin{itemize}[noitemsep]
      \item \href{https://githubcom/Zeta611/modern-tex-engines-automation-ai-latex-workshop-2023}{Modern \TeX: Engines, AI, and Automation}, 2023 KNU \LaTeX\ Workshop
      \item \href{https://github.com/Zeta611/tabularray-tutorial-latex-workshop-2022}{Drawing tables with \texttt{tabularray}}, 2022 KNU \LaTeX\ Workshop
      \item \href{https://github.com/Zeta611/asymptote-tutorial-latex-workshop-2021}{Asymptote: The Vector Graphics Language}, 2022 KNU \LaTeX\ Workshop
      \item \href{https://github.com/Zeta611/key-value-tutorial-latex-workshop-2021}{The ``key-value'' structure in \LaTeX}, 2021 KNU \LaTeX\ Workshop
      \item \href{https://github.com/Zeta611/beamer-tutorial-latex-workshop-2020}{\texttt{beamer}: Content-focused Presentation}, 2020 KNU \LaTeX\ Workshop
      \item \href{https://github.com/Zeta611/texnical-vim-kts-conf-2020}{\TeX{}nical Vim}, 2020 KTS Conference
      \item \href{https://github.com/Zeta611/chapterstyle-latex-workshop-2019}{\texttt{memoir}: chapter style}, 2019 KNU \LaTeX\ Workshop
    \end{itemize}
    \vspace{-0.5\onelineskip}
  },
}
\vspace{0.5\onelineskip}

\Entry{
  date     = {2018--2024},
  title    = {SNULife},
  position = {Development head},
  body     = {SNULife is the web community for SNU with 180k+ monthly visitors. I created an iOS app for planning timetables and sharing lecture reviews, used by 5k+ users.},
  url      = {https://www.snulife.com/},
}
\vspace{0.5\onelineskip}

\Entry{
  date     = {2020--2021},
  title    = {Republic of Korea (ROK) Army},
  position = {Sergeant signaller},
  body     = {Completed mandatory military service in the ROK Army.},
}
\vspace{0.5\onelineskip}

\Entry{
  date     = {2018},
  title    = {Student Representative},
  position = {Department of Electrical and Computer Engineering},
  body     = {Elected as a student representative for the Department of Electrical and Computer Engineering at SNU.},
}
\vspace{0.5\onelineskip}

\Entry{
  date     = {KYPT 2016, 2017},
  title    = {Korea Young Physicists' Tournament},
  position = {Team lead},
  body     = {KYPT is a physics competition for high school students. I led the team that won a gold medal in KYPT 2017 and a bronze medal in KYPT 2016.},
}
\vspace{0.5\onelineskip}

\Entry{
  date     = {KIPS 2016},
  title    = {Receding Horizon Next-Best-View Planner Based Voronoi-Biased 3D Multi-Robot Exploration Algorithm},
  body     = {\ul{J. Lee}, C. Lee, W. Jung, S. Song, S. Jo},
  note     = {A domestic conference paper.},
  paper    = {https://doi.org/10.3745/PKIPS.y2016m10a.579},
}
\vspace{0.5\onelineskip}

\Entry{
  date     = {2016},
  title    = {Frontiers Summer Program},
  position = {Worcester Polytechnic Institute},
  body     = {Studied aerospace engineering and psychology at Worcester Polytechnic Institute in the USA.},
}

\section{Natural Languages}
% \worldflag[width=1em]{KR} \textbf{Korean} (native), \worldflag[width=1em]{GB} \textbf{English} (fluent), \worldflag[width=1em]{ES} \textbf{Spanish} (elementary)
\textbf{Korean/한국어}~(native, 1999), \textbf{English}~(fluent, 2006), \textbf{Spanish/Español}~(elementary, 2022)

\section{Programming Languages}
C~(2012), \textbf{Python}~(2013), \C++~(2016), \textbf{\TeX}~(2016), Swift~(2018), Ti\textit{k}Z~(2018), \textbf{OCaml}~(2019), \lambda{} calculus~(2019), \textbf{\LaTeX3/expl3}~(2019), Asymptote~(2020), Scheme~(2020), AWK~(2020), JavaScript~(2020), Lua~(2021), \texttt{CWEB}~(2021), Yacc~(2021), \textbf{ReScript}~(2022), \textbf{React}~(2022), Rocq/Coq~(2023), Rust~(2023), \textbf{TypeScript}~(2023), Typst~(2024), Penrose~(2024), Nix~(2024), Lean~(2025).
I use the \textbf{bolded} languages frequently.

\vfill\hfill \textit{Last updated on \today~(using \LaTeX3{} with Neovim on NixOS \& macOS btw).}
\end{document}
